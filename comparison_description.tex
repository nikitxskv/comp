\documentclass{article}
\usepackage[utf8]{inputenc}
\usepackage[russian]{babel}
\usepackage{vmargin}
\setpapersize{A4}
\setmarginsrb{2cm}{1.5cm}{1cm}{1.5cm}{0pt}{0mm}{0pt}{13mm}
\usepackage{indentfirst}
\usepackage{amsmath}
\usepackage{amsfonts,amssymb}
\usepackage{amsthm}
\usepackage{graphicx}
\usepackage{float}
\usepackage[usenames]{color}
\usepackage{colortbl}
\usepackage{xcolor}
\colorlet{linkequation}{red}
\usepackage[colorlinks]{hyperref}
\usepackage{cancel}
\usepackage{tcolorbox}
\usepackage{graphicx}
\usepackage{mathtools}
\usepackage{multicol}
\usepackage{paralist}

\newcommand{\argmin}{\mathop{\rm arg\,min}\limits}
\newcommand{\argmax}{\mathop{\rm arg\,max}\limits}
\newcommand{\sign}{\mathop{\rm sign}\limits}
\newcommand{\cond}{\mspace{3mu}{|}\mspace{3mu}}

\def\RR{\mathbb{R}}
\def\XX{\mathbb{X}}
\def\EE{\mathbb{E}}
\def\NN{\mathcal{N}}
\def\LL{\mathcal{L}}
\def\YY{\mathbb{Y}}
\def\OO{\mathcal{O}}

\sloppy

\title{Описание формата сравнения.}
\date{Апрель, 2017}

\begin{document}

\maketitle
\noindent\textbf{Входные данные.}
\medskip

Все эксперименты проводились на следующих датасетах.

Для задач классификации:

\noindent Adult, amazon, appet, click, criteo, internet, kdd98, kddchurn, kick, paribas, springleaf, upsel.

Для задач регрессии:

\noindent Allstate, bimbo, liberty.

Все датасеты разбивались на обучающую и тестовую части в соотношении 4:1 соответственно. Обозначим их за $(X_{full\_train}, y_{full\_train})$ и $(X_{test}, y_{test})$.

\medskip
\noindent\textbf{Предобработка датасета.}
\medskip

Итак, на входе имеется обучающая $(X_{full\_train}, y_{full\_train})$ и тестовая $(X_{test}, y_{test})$ выборки, а также список номеров колонок категориальных признаков.

В экспериментах используется 5-фолдовая кросс-валидация. Поэтому $(X_{full\_train}, y_{full\_train})$ разбивается на 5 подвыборок $(X_1, y_1), \dots, (X_5, y_5)$, и из них конструируется 5 выборок вида $(X^{train}_i, y^{train}_i)$, $(X^{val}_i, y^{val}_i)$ таким образом, что $(X^{val}_i, y^{val}_i)$ совпадает с $(X_i, y_i)$, а $(X^{train}_i,y^{train}_i)$ совпадает с $\cup_{j\neq i}(X_j, y_j)$.

Далее, для каждой такой пары, мы предобрабатываем категориальные признаки по следующей схеме.

Пусть имеется обучающая $(X^{train}, y^{train})$ и валидационная $(X^{val}, y^{val})$ выборки. Для простоты обозначений будем считать, что все признаки категориальные. Вводим понятие времени в обучающей выборке.
На выборках, в которых рисутствует признак "время" \--- упорядочиваем все по нему, если же такого признака нет, то производим случайную перестановку объектов. Считаем, что для задач классификации метки классов принадлежат множеству \{0, 1\}. Далее, для каждого j-го признака и i-го объекта, считаются 2 числа $a_{ij}$ и $b_{ij}$:

$$a_{ij} = \sum_{k=1}^{i - 1}[X^{train}_{ij} = X^{train}_{kj}]y^{train}_{kj},$$

$$b_{ij} = \sum_{k=1}^{i - 1}[X^{train}_{ij} = X^{train}_{kj}] \text{, где } [\dots] \text{- индикатор.}$$

Теперь в обучающей выборке категориальные признаки заменяются на числовые по следующей формуле.

\textbf{Для задач классификации:}

$$X^{train}_{ij} = \frac{a_{ij} + 1}{b_{ij} + 2}.$$

\textbf{Для задач регрессии:}

$$X^{train}_{ij} = \begin{cases} \frac{a_{ij}}{b_{ij}}, & \mbox{if } b_{ij} \neq 0  \\ 0, & \mbox{if } b_{ij} = 0 \end{cases}.$$

Далее необходимо заменить категориальные признаки в валидационной выборке. Для этого, для каждого j-го признака и i-го объекта так же считаются 2 числа $c_{ij}$ и $d_{ij}$:

$$c_{ij} = \sum_{k}[X^{val}_{ij} = X^{train}_{kj}]y^{train}_{kj},$$

$$d_{ij} = \sum_{k}[X^{val}_{ij} = X^{train}_{kj}] \text{, где } [\dots] \text{- индикатор.}$$

Теперь в валидационной выборке категориальные признаки заменяются на числовые по следующей формуле.

\textbf{Для задач классификации:}

$$X^{val}_{ij} = \frac{c_{ij} + 1}{d_{ij} + 2}.$$

\textbf{Для задач регрессии:}

$$X^{val}_{ij} = \begin{cases} \frac{c_{ij}}{d_{ij}}, & \mbox{if } d_{ij} \neq 0  \\ 0, & \mbox{if } d_{ij} = 0 \end{cases}.$$


Таким образом получилось 5 пар (обучающая и валидационная) выборок, которые содержат только числовые значения.

Далее, для исходных выборок $(X_{full\_train}, y_{full\_train})$ и $(X_{test}, y_{test})$, также заменим категориальные признаки на числовые по той же самой схеме, что и для $(X^{train}, y^{train})$ и $(X^{val}, y^{val})$.

\medskip
\noindent\textbf{Сетка параметров.}
\medskip

Параметры подбираются с помощью библиотеки hyperopt. Ниже приведен список подбираемых параметров и распределений, откуда они выбирались для каждого алгоритма:

\medskip
\noindent XGBoost.
\begin{itemize}
  \item 'eta': Логравномерное распределение $[e^{-7}, 1]$
  \item 'max\_depth' : Дискретное равномерное распределение $[2, 10]$
  \item 'subsample': Равномерное $[0.5, 1]$
  \item 'colsample\_bytree': Равномерное $[0.5, 1]$
  \item 'colsample\_bylevel': Равномерное $[0.5, 1]$
  \item 'min\_child\_weight': Логравномерное распределение $[e^{-16}, e^{5}]$
  \item 'alpha': Смесь: 0.5 * Вырожденное в 0 + 0.5 * Логравномерное распределение $[e^{-16}, e^{2}]$
  \item 'lambda': Смесь: 0.5 * Вырожденное в 0 + 0.5 * Логравномерное распределение $[e^{-16}, e^{2}]$
\end{itemize}

\medskip
\noindent LightGBM.
\begin{itemize}
  \item 'learning\_rate': Логравномерное распределение $[e^{-7}, 1]$
  \item 'num\_leaves' : Дискретное логравномерное распределение $[1, e^{7}]$
  \item 'feature\_fraction': Равномерное $[0.5, 1]$
  \item 'bagging\_fraction': Равномерное $[0.5, 1]$
  \item 'min\_sum\_hessian\_in\_leaf': Логравномерное распределение $[e^{-16}, e^{5}]$
  \item 'min\_data\_in\_leaf':  Дискретное логравномерное распределение $[1, e^{6}]$
  \item 'lambda\_l1': Смесь: 0.5 * Вырожденное в 0 + 0.5 * Логравномерное распределение $[e^{-16}, e^{2}]$
  \item 'lambda\_l2': Смесь: 0.5 * Вырожденное в 0 + 0.5 * Логравномерное распределение $[e^{-16}, e^{2}]$
  \item 'max\_bin': Дискретное логравномерное распределение $[1, e^{20}]$
\end{itemize}

\medskip
\noindent\textbf{Подбор параметров.}
\medskip

При подборе, в каждом алгоритме выставляется параметр, отвечающий за максимальное число деревьев, равный 2000. Далее, для каждого конкретного набора параметров, в каждом из 5 фолдов, при добавлении очередного дерева в алгоритм, подсчитываются значения метрик на валидационной выборке. В итоге получается 5 2000-мерных векторов, которые усредняются в один вектор, по которому берется аргмаксимум. Полученное число и является оптимальным количеством деревьев для данного набора параметров.

В итоге, производилось 50 итераций подбора параметров и выбирались те параметры, на которых получалась наилучшая метрика LogLoss.


\medskip
\noindent\textbf{Итоговый результат.}
\medskip

В итоге, в алгоритме выставляются оптимальные параметры, и запускается обучение на предобработанном $(X_{full\_train}, y_{full\_train})$. После этого вычисляется значение метрики LogLoss на предобработанной тестовой выборке $(X_{test}, y_{test})$.

\medskip
\noindent\textbf{Версии библиотек.}

\begin{itemize}
  \item xgboost (0.6)
  \item scikit-learn (0.18.1)
  \item scipy (0.19.0)
  \item pandas (0.19.2)
  \item numpy (1.12.1)
  \item lightgbm (0.1)
  \item hyperopt (0.0.2)
\end{itemize}


\end{document}
